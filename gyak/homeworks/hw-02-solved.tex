\documentclass[11pt, a4paper]{article}

%%%%%%%%%%%%%%%%%
% Configuration %
%%%%%%%%%%%%%%%%%
\usepackage{allrunes}
\usepackage{amsmath}
% If magyar is wanted
% \usepackage[magyar]{babel}
\usepackage[T1]{fontenc}
\usepackage[utf8]{inputenc}
\usepackage{fixltx2e}
\usepackage{multirow}
\usepackage{url}
\usepackage{amsfonts}
\usepackage{amsthm}
\usepackage{mathtools}
\usepackage{amssymb}
\usepackage{xcolor}

% using circled symbols
\usepackage{tikz}
\newcommand*\circled[1]{\tikz[baseline=(char.base)]{
            \node[shape=circle,draw,inner sep=2pt] (char) {#1}}}


% Here you can configure the layout
\usepackage{geometry}
\geometry{top=1cm, bottom=1cm, left=1.25cm,right=1.25cm, includehead, includefoot}
\setlength{\columnsep}{7mm} % Column separation width

\usepackage{graphicx}

%\usepackage{gensymb}
\usepackage{float}

% For bra-ket notation
\usepackage{braket}

% To have a good appendix
\usepackage[toc,page]{appendix}

\usepackage{abstract}
\renewcommand{\abstractnamefont}{\normalfont\bfseries}
\renewcommand{\abstracttextfont}{\normalfont\small\itshape}
\usepackage{lipsum}

%%%%%%%%%%%%%%%%%%%
% Custom commands %
%%%%%%%%%%%%%%%%%%%
\newcommand{\bb}[1]{\mathbf{#1}}
\newcommand{\dd}{\mathrm{d}}
\newcommand{\Tr}[1]{\mathrm{Tr}\left[#1\right]}
\newcommand{\Sp}[1]{\mathrm{Sp}\left[{#1}\right]}

% \newtheorem*{tetel*}{Tétel}
% \newtheorem*{defn*}{Definíció}
% \newtheorem*{pld*}{Példa}
% \newtheorem*{megj*}{Megjegyzés}
% \newtheorem*{allit*}{Állítás}

% \newtheorem{tetel}{Tétel}
% \newtheorem{defn}{Definíció}
% \newtheorem{pld}{Példa}
% \newtheorem{megj}{Megjegyzés}
% \newtheorem{allit}{Állítás}

% Hyperref should be generally the last package to load
% Any configuration that should be done before the end of the preamble:

\usepackage{hyperref}
\hypersetup{colorlinks=true, urlcolor=blue, linkcolor=blue, citecolor=blue}

\title{Statistical physics cheat sheet}

\author{Nagy Dániel}
\date{\today}

\begin{document}
\maketitle
\newpage

\begin{enumerate}
    %%%%%%%%%%%%%%%%%%%%%%%%%%%%%%%%%%%%%%%%%%%%%%%%%%%%%%%%%%%%%%%%%%%%%%%%%%%%%%%%%%%%%%%%%%%%%%%
                                            % EXERCISE 1 %
    %%%%%%%%%%%%%%%%%%%%%%%%%%%%%%%%%%%%%%%%%%%%%%%%%%%%%%%%%%%%%%%%%%%%%%%%%%%%%%%%%%%%%%%%%%%%%%%
    \item Using the second quantized formalism, show that for noninteracting fermions
    \begin{equation*}
        \Omega_0 = -k_BT\sum\limits_i \ln \left(1 + e^{-\beta(\varepsilon_i - \mu)}\right)\,,
    \end{equation*}
    where $\varepsilon_i$ denotes the $i$-th one-particle level
    
    %%%%%%%%%%%%%%%%%%%%%%%%%%%%%%%%%%%%%%%%%%%%%%%%%%%%%%%%%%%%%%%%%%%%%%%%%%%%%%%%%%%%%%%%%%%%%%%
                                            % EXERCISE 2 %
    %%%%%%%%%%%%%%%%%%%%%%%%%%%%%%%%%%%%%%%%%%%%%%%%%%%%%%%%%%%%%%%%%%%%%%%%%%%%%%%%%%%%%%%%%%%%%%%
    \item Using the result for $\Omega_0$, calculate $\Omega_0$ and $N$ as a function of $(T,V,\mu)$
    for a fermionic homogeneous system (noninteracting fermions in a box with periodic boundary
    conditions). Express your result with Fermi-Dirac integrals. Give the first three terms of the
    high temperature expansion for $\Omega_0$ and for $N$.
    
    %%%%%%%%%%%%%%%%%%%%%%%%%%%%%%%%%%%%%%%%%%%%%%%%%%%%%%%%%%%%%%%%%%%%%%%%%%%%%%%%%%%%%%%%%%%%%%%
                                            % EXERCISE 3 %
    %%%%%%%%%%%%%%%%%%%%%%%%%%%%%%%%%%%%%%%%%%%%%%%%%%%%%%%%%%%%%%%%%%%%%%%%%%%%%%%%%%%%%%%%%%%%%%%
    \item For noninteracting fermions, one can define a characteristic temperature $T_{\textrm{deg}}$
    by which the chemical potential is zero:
    \begin{equation*}
        \mu(T=T_{\textrm{deg}}) = 0.
    \end{equation*}
    By dimensional analysis
    \begin{equation*}
        k_BT_{\textrm{deg}} = z \frac{\hbar^2}{2m}\left(\frac{N}{V}\right)^{2/3}\,,
    \end{equation*}
    where $z$ is a dimensionless number. Calculate $z$ exactly and numerically.

    %%%%%%%%%%%%%%%%%%%%%%%%%%%%%%%%%%%%%%%%%%%%%%%%%%%%%%%%%%%%%%%%%%%%%%%%%%%%%%%%%%%%%%%%%%%%%%%
                                            % EXERCISE 4 %
    %%%%%%%%%%%%%%%%%%%%%%%%%%%%%%%%%%%%%%%%%%%%%%%%%%%%%%%%%%%%%%%%%%%%%%%%%%%%%%%%%%%%%%%%%%%%%%%
    \item Let us suppose that we have $N$ noninteracting, spinless bosons confined in a
    3 dimensional harmonic oscillator potential
    \begin{equation*}
        V(\mathbf{r}) = \frac{m}{2}(\omega_1^2x^2 + \omega_2^2y^2 + \omega_3^2z^2).
    \end{equation*}
    Calculate $T_c$, where Bose-Einstein condensation occurs.
\end{enumerate}
\end{document}