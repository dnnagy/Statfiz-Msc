\documentclass[11pt, a4paper]{article}

%%%%%%%%%%%%%%%%%
% Configuration %
%%%%%%%%%%%%%%%%%
\usepackage{allrunes}
\usepackage{amsmath}
% If magyar is wanted
% \usepackage[magyar]{babel}
\usepackage[T1]{fontenc}
\usepackage[utf8]{inputenc}
\usepackage{fixltx2e}
\usepackage{multirow}
\usepackage{url}
\usepackage{amsfonts}
\usepackage{amsthm}
\usepackage{mathtools}
\usepackage{amssymb}
\usepackage{xcolor}

% using circled symbols
\usepackage{tikz}
\newcommand*\circled[1]{\tikz[baseline=(char.base)]{
            \node[shape=circle,draw,inner sep=2pt] (char) {#1}}}


% Here you can configure the layout
\usepackage{geometry}
\geometry{top=1cm, bottom=1cm, left=1.25cm,right=1.25cm, includehead, includefoot}
\setlength{\columnsep}{7mm} % Column separation width

\usepackage{graphicx}

%\usepackage{gensymb}
\usepackage{float}

% For bra-ket notation
\usepackage{braket}

% To have a good appendix
\usepackage[toc,page]{appendix}

\usepackage{abstract}
\renewcommand{\abstractnamefont}{\normalfont\bfseries}
\renewcommand{\abstracttextfont}{\normalfont\small\itshape}
\usepackage{lipsum}

%%%%%%%%%%%%%%%%%%%
% Custom commands %
%%%%%%%%%%%%%%%%%%%
\newcommand{\bb}[1]{\mathbf{#1}}
\newcommand{\dd}{\mathrm{d}}
\newcommand{\Tr}[1]{\mathrm{Tr}\left[#1\right]}
\newcommand{\Sp}[1]{\mathrm{Sp}\left[{#1}\right]}

% \newtheorem*{tetel*}{Tétel}
% \newtheorem*{defn*}{Definíció}
% \newtheorem*{pld*}{Példa}
% \newtheorem*{megj*}{Megjegyzés}
% \newtheorem*{allit*}{Állítás}

% \newtheorem{tetel}{Tétel}
% \newtheorem{defn}{Definíció}
% \newtheorem{pld}{Példa}
% \newtheorem{megj}{Megjegyzés}
% \newtheorem{allit}{Állítás}

% Hyperref should be generally the last package to load
% Any configuration that should be done before the end of the preamble:

\usepackage{hyperref}
\hypersetup{colorlinks=true, urlcolor=blue, linkcolor=blue, citecolor=blue}

\title{Statistical physics cheat sheet}

\author{Nagy Dániel}
\date{\today}

\begin{document}
\maketitle
\newpage

\section{Fock-states}
\begin{align*}
    &\ket 0 = \ket{0,0,0,\dots} \\
    &\braket{0|0} = \braket{\dots,0,0,0|0,0,0,\dots} = 1\\
    &\braket{\dots, n_i', \dots, n_2', n_1' | n_1, n_2, \dots, n_i, \dots}
    = \cdots\times\delta_{n_1n_1'}\times\delta_{n_2n_2'}\times\cdots\times\delta_{n_in_i'}\times\cdots
\end{align*}

\section{Creation and annihillation operators}
\subsection{Fermionic creation and annihillation operators}
\begin{align*}
    &\hat a_k^{\dagger} \ket{n_1, \dots, n_k, \dots} = \sqrt{1-n_k} (-1)^{\Sigma_k} \ket{n_1, \dots, 1 + n_k, \dots} \\
    &\hat a_k \ket{n_1, \dots, n_k, \dots} = \sqrt{n_k} (-1)^{\Sigma_k} \ket{n_1, \dots, 1 - n_k, \dots}\\
    &n_k \in \{0,1\},~ \Sigma_k = \sum\limits_{j=1}^{k-1}n_j \\
    &\hat a_k^{\dagger} \ket{n_1, \dots, n_{k-1}, 1, n_{k+1},\dots} = 0 \\
    &\hat a_k \ket{n_1, \dots, n_{k-1}, 0, n_{k+1},\dots} = 0\\
    &\hat a_k = (\hat a_k^{\dagger})^{\dagger}
\end{align*}
Anticommutation relations:
\begin{align*}
    &\{\hat a_k, \hat a_l^{\dagger}\}  = \hat a_k\hat a_l^{\dagger}+\hat a_l^{\dagger}\hat a_k = \delta_{kl}\\
    &\{\hat a_k, \hat a_l\} = \{ \hat a_k^{\dagger}, \hat a_l^{\dagger} \} = 0
\end{align*}
\subsection{Bosonic creation and annihillation operators}
\begin{align*}
    &\hat a_k^{\dagger} \ket{n_1, \dots, n_k, \dots} = \sqrt{1+n_k}\ket{n_1, \dots, n_k + 1, \dots} \\
    &\hat a_k \ket{n_1, \dots, n_k, \dots} = \sqrt{n_k} \ket{n_1, \dots, n_k - 1, \dots} \\
    &n_k \in \{0,1,\dots\}\\
    &\hat a_k^{\dagger} \ket{n_1, \dots, n_{k-1}, 1, n_{k+1},\dots} = 0 \\
    &\hat a_k \ket{n_1, \dots, n_{k-1}, 0, n_{k+1},\dots} = 0\\
\end{align*}
Commutation relations:
\begin{align*}
    &[\hat a_k, \hat a_l^{\dagger}]  = \hat a_k\hat a_l^{\dagger}+\hat a_l^{\dagger}\hat a_k = \delta_{kl}\\
    &[\hat a_k, \hat a_l] = [ \hat a_k^{\dagger}, \hat a_l^{\dagger} ] = 0
\end{align*}
For bosons only:
\begin{align*}
    &\ket{n_1, \dots, n_k, \dots} = \frac{1}{\sqrt{\prod\limits_{i=1}^{\infty}n_i!}}
    (\hat a_1^{\dagger})^{n_1}\times(\hat a_2^{\dagger})^{n_2}\times\dots\times(\hat a_k^{\dagger})^{n_k}\times
    \dots\ket{0}.
\end{align*}
\textbf{For both bosons and fermions:}
\begin{itemize}
    \item $\hat n_k = \hat a_k^{\dagger}\hat a_k $ gives the number of particles in the $k$-th eigenstate:
    \begin{equation*}
        \hat n_k \ket{n_1, \dots, n_k, \dots} = n_k \ket{n_1, \dots, n_k, \dots}.
    \end{equation*}
    \item $\hat N = \sum\limits_k \hat n_k$ gives the total number of particles:
    \begin{equation*}
        \hat N \ket{n_1, \dots, n_k, \dots} = \sum\limits_{j=1}^{\infty}n_j \ket{n_1, \dots, n_k, \dots}
        = N \ket{n_1, \dots, n_k, \dots}.
    \end{equation*}
    \item $\braket{0|0} = 1$
    \item $\bra 0 \hat a_k^{\dagger} = 0$
    \item $\hat a_k \ket 0 = 0$
\end{itemize}

\section{Field operators}
\begin{align*}
    &\hat\Psi^{\dagger}(x) = \sum\limits_{j=1}^{\infty}\varphi^*_j(x)\hat a_j^{\dagger}\\
    &\hat\Psi(x) = \sum\limits_{j=1}^{\infty}\varphi_j(x)\hat a_j,
\end{align*}
where $\varphi_j(x) = \braket{x|j}$ is the wavefunction of the $j$-th one-particle eigenstate.

The field operator satisfy the anticommutation relations for fermions:
\begin{align*}
    &\{\hat\Psi(x),\hat\Psi^{\dagger}(y)\} = \delta(x-y)\\
    &\{\hat\Psi^{\dagger}(x),\hat\Psi^{\dagger}(y)\} = \{\hat\Psi(x),\hat\Psi(y)\} = 0
\end{align*}
And for bosons, the commutation relations:
\begin{align*}
    &[\hat\Psi(x),\hat\Psi^{\dagger}(y)] = \delta(x-y)\\
    &[\hat\Psi^{\dagger}(x),\hat\Psi^{\dagger}(y)] = [\hat\Psi(x),\hat\Psi(y)] = 0
\end{align*}

The particle number operator is 
\begin{equation*}
    \hat N = \int\dd x \hat\Psi^{\dagger}(x)\hat\Psi(x) = \dots = \sum\limits_{k=1}^{\infty}\hat a_k^{\dagger}\hat a_k
\end{equation*}

Calculating the first-quantized wavefunction for an $N$-particle system:
Let $\ket{\Phi_N}$ be a second quantized state for which
\begin{equation*}
    \hat N \ket{\Phi_N} = N \ket{\Phi_N}.
\end{equation*}
The first-quantized wavefunction for this state is 
\begin{equation*}
    \Phi(x_1, x_2, \dots, x_N) = \frac{1}{\sqrt{N!}}\Braket{0| \prod\limits_{j=1}^{N} \hat\Psi(x_j) | \Phi_N }.
\end{equation*}

\section{Fock-space operators}
If $\hat o$ is a single-particle operator in first-quantization, then its second quantized form is 
\begin{align*}
    \hat O &= \int\dd x\hat\Psi^{\dagger}(x)\hat o\hat\Psi(x) \\
    & = \sum\limits_{j,k} \int \dd x \varphi^*_j(x) \hat o \varphi_k(x) \hat a_j^{\dagger}\hat a_k.
\end{align*} 
If $\hat o$ is a two-particle operator in first-quantization, then its second-quantized form is 
\begin{align*}
    \hat O &= \frac{1}{2} \int\dd x \dd x' \hat\Psi^{\dagger}(x) \hat\Psi^{\dagger}(x') \hat o \hat\Psi(x') \hat\Psi(x) \\
    & = \frac{1}{2} \sum\limits_{i,j,k,l} \int\dd x \dd x' \varphi^*_i(x)\varphi^*_j(x') \hat o \varphi_k(x')\varphi_l(x)
    \hat a_i^{\dagger}\hat a_j^{\dagger}\hat a_k\hat a_l.
\end{align*}


\section{Green function method}
\subsection{Grand canonical ensemble}
\begin{align*}
    &\hat H = \hat H_0 + \hat H_1\\
    &\hat K \vcentcolon= \hat H - \mu \hat N\\
    &\hat K = \hat K_0 + \hat K_1\\
    &\hat K_0 = \hat H_0 - \mu \hat N, ~ \hat K_1 = \hat H_1
\end{align*}
The trace of an operator:
\begin{equation*}
    \Tr{\hat A} = \Tr{\hat A} = \sum\limits_{\{n_i\}}\braket{\dots, n_i, \dots, n_1 | \hat A |n_1, \dots, n_i, \dots},
\end{equation*}
where the sum is over the entire Fock-space, not just the $N$-particle subspace.
\par The grand canonical partition function is defined as 
\begin{equation*}
    Z_G = e^{-\beta\Omega(T,V,\mu)} = \Tr{e^{-\beta\hat K}} = 
    \sum\limits_{\{n_i\}}\braket{\dots, n_i, \dots, n_1 | e^{-\beta\hat K} |n_1, \dots, n_i, \dots},
\end{equation*}
where $\beta = (1/k_BT)$.
\par The grand canonical density matrix is 
\begin{equation*}
    \hat\rho_G = \frac{e^{-\beta\hat K}}{Z_G}.
\end{equation*}
The average of operator $\hat O$ \textit{over the grand canonical ensemble} is
\begin{equation*}
    \langle \hat O \rangle = \Tr{\hat \rho_G \hat O} = \frac{1}{Z_G}\Tr{e^{-\beta \hat K}\hat O}.
\end{equation*}


\section{Exercises}
\begin{enumerate}
    \item What is the first quantized wavefunction of $\sum\limits_k c_k \hat a_k^{\dagger}\ket 0$, where
    $\hat a_k^{\dagger}$ is a fermionic creation operator?
    \item Calculate the first-quantized wavefunction for $\hat a_k^{\dagger}\hat a_l^{\dagger}\ket 0$ for
    both fermionic and bosonic operators!
    \item Consider 2 fermionic particles prepared in states
    \begin{align*}
        &\Psi_1(x_1) = \sum\limits_k b_k\varphi_k(x_1)\\
        &\Psi_2(x_2) = \sum\limits_l c_l\varphi_l(x_2).
    \end{align*}
    The state of the joint system in first quantization is 
    \begin{equation*}
        \Phi(x_1, x_2) = \frac{1}{\sqrt 2}[\Psi_1(x_1)\Psi_2(x_2) - \Psi_1(x_2)\Psi_2(x_1)]
    \end{equation*}
    What is the Fock-space representation of $\Phi(x_1, x_2)$? Verify Your result by converting it back to 
    first-quantization using the formula!
    \par \textit{Solution.} 
    \begin{align*}
        &\Psi_1(x_1) = \sum\limits_k b_k\varphi_k(x_1) \rightarrow \sum\limits_k b_k \hat a_k^{\dagger}\ket{0} \\
        &\Psi_2(x_2) = \sum\limits_l c_l\varphi_l(x_2) \rightarrow \sum\limits_l c_l \hat a_k^{\dagger}\ket{0} \\
        &\Phi(x_1, x_2) = \sum\limits_{k,l} b_kc_l \frac{1}{\sqrt 2} [\varphi_k(x_1)\varphi_l(x_2) - \varphi_k(x_2)\varphi_l(x_1)]
        \rightarrow \sum\limits_{k,l} b_kc_l \hat a_k^{\dagger}a_l^{\dagger} \ket 0 
    \end{align*}
    \item Find the eigenvalues and eigenvectors of $\hat a_k$ and $\hat a_k^{\dagger}$!
    \item $\hat\Psi^{\dagger}(x)\ket 0 = ?$
    \item $[\hat \Psi(x), \hat N] = ?$
    \item Calculate the first-quantized wavefunction $\Phi(x_1,x_2,x_3)$ for bosonic three-particle system 
    prepared in state $\ket{2, 0, 1, 0, 0, \cdots}$.
    \item Prove that 
    \begin{align*}
        &[\hat a_j, f(\hat a_j^{\dagger})] = \frac{\partial f(\hat a_j^{\dagger})}{\partial \hat a_j^{\dagger}}\\
        &[\hat a_j^{\dagger}, f(\hat a_j)] = -\frac{\partial f(\hat a_j)}{\partial \hat a_j}
    \end{align*}
    Hint: Use the Taylor-expansion formula:
    \begin{equation*}
        f(\hat A) = \sum\limits_{k=1}^{\infty} \frac{1}{k!}f^{(k)}(0)\hat A^k
    \end{equation*}
    And the derivative is 
    \begin{equation*}
        \frac{\partial f(\hat A)}{\partial A} = \sum\limits_{k=1}^{\infty} \frac{1}{(k-1)!}f^{(k)}(0)\hat A^{k-1}
    \end{equation*}
    \item Consider a fermionic system. What is the Fock-space representation of $\hat S_z$ and what is the meaning
    of this operator? What are the eigenstates of $\hat S_z$?
    \item Calculate $\Tr {\hat N}$ for bosonic particles!
    \par\textit{Solution}.
    \begin{align*}
        \Tr{\hat N} &= \sum\limits_{\{n_i\}}\braket{\dots,n_i,\dots,n_1 | \hat N |n_1, \dots, n_i, \dots} \\
        & = \sum\limits_{\{n_i\}}\Braket{\dots,n_i,\dots,n_1 | \int\dd x \hat\Psi^{\dagger}(x)\hat\Psi(x) |n_1, \dots, n_i, \dots} \\
        & = \sum\limits_{\{n_i\}}\Braket{\dots,n_i,\dots,n_1 | \int\dd x \sum\limits_{k,l}\varphi^*_k(x)\varphi_l(x)
        \hat a_k^{\dagger}\hat a_l |n_1, \dots, n_i, \dots} \\
        & = \sum\limits_{\{n_i\}}\sum\limits_{k,l}\int\dd x \varphi^*_k(x)\varphi_l(x) \braket{\dots,n_i,\dots,n_1 | \hat a_k^{\dagger}\hat a_l |n_1, \dots, n_i, \dots} \\
        & = \sum\limits_{\{n_i\}}\sum\limits_{k,l}\int\dd x \varphi^*_k(x)\varphi_l(x) \braket{\dots,n_i,\dots,n_1 | \hat a_k^{\dagger}
        \sqrt{n_l} |n_1, \dots, n_l-1, \dots} \\
        & = \sum\limits_{\{n_i\}}\sum\limits_{k,l}\int\dd x \varphi^*_k(x)\varphi_l(x) \braket{\dots,n_i,\dots,n_1 | 
        \sqrt{n_k+1}\sqrt{n_l} |n_1, \dots, n_k+1, \dots, n_l-1, \dots} \\
        & = \sum\limits_{\{n_i\}}\sum\limits_{k,l} \sqrt{n_k+1}\sqrt{n_l} \int\dd x \varphi^*_k(x)\varphi_l(x) \braket{\dots,n_i,\dots,n_1 |n_1, \dots, n_k+1, \dots, n_l-1, \dots} \\
        & = \sum\limits_{\{n_i\}}\sum\limits_{k,l} \sqrt{n_k+1}\sqrt{n_l} \int\dd x \varphi^*_k(x)\varphi_l(x) \delta_{kl} \\
        & = \sum\limits_{\{n_i\}}\sum\limits_{k} n_k \underbrace{\int\dd x \varphi^*_k(x)\varphi_k(x)}_{=1}\\
        & = \sum\limits_{\{n_i\}}\sum\limits_{k} n_k
    \end{align*}
    \item Calculate $\Tr {\hat H_0}$, where
    \begin{equation*}
        \hat H_0 = \int\dd x \hat\Psi^{\dagger}(x)\left[ -\frac{\hbar^2}{2M}\nabla^2 + U(x) \right]\hat\Psi(x)
    \end{equation*}
    \item Calculate $\langle \hat n_k \rangle$ for free non-interacting particles!
    \par \textit{Solution.}
    \begin{equation*}
        \langle \hat n_k \rangle = \langle \hat a_k^{\dagger}\hat a_k \rangle
        = \frac{1}{Z_G}\Tr{e^{-\beta \hat K_0}\hat a_k^{\dagger}\hat a_k} 
        = \frac{1}{Z_G}\Tr{\hat a_ke^{-\beta \hat K_0}\hat a_k^{\dagger}}
        = \frac{1}{Z_G}\Tr{e^{-\beta \hat K_0}e^{\beta \hat K_0}\hat a_ke^{-\beta \hat K_0}\hat a_k^{\dagger}}.
    \end{equation*}
    We can use the imaginary time-dependence of operators ($K$-picture):
    \begin{align*}
        &\hat O_K(\tau) = e^{\frac{\hat K\tau}{\hbar}}\hat Oe^{-\frac{\hat K\tau}{\hbar}} \\
        &\frac{\dd}{\dd\tau} \hat O_K(\tau) = \frac{1}{\hbar}[\hat K, \hat O_K(\tau)]
    \end{align*}
    \begin{align*}
        \hat a_k(\tau) &= e^{\frac{\hat K\tau}{\hbar}}\hat a_k e^{-\frac{\hat K\tau}{\hbar}}
        \Longrightarrow
        \hat a_k(\beta\hbar) = e^{\beta \hat K_0}\hat a_ke^{-\beta \hat K_0}\\
        \frac{\dd}{\dd\tau}\hat a_k(\tau) &= \frac{1}{\hbar}[\hat K_0, \hat a_k(\tau)]\\
        & = \frac{1}{\hbar}\left[\sum\limits_{l,m}\int\dd x \varphi^*_l(x) \left(-\frac{\hbar^2}{2M}\nabla^2 + U(x) - \mu\right)
        \varphi_m(x)\hat a_l^{\dagger}\hat a_m, \hat a_k(\tau) \right] \\
        & = \frac{1}{\hbar}\left[\sum\limits_{l}a_l^{\dagger}\hat a_l (e_l - \mu), \hat a_k(\tau)\right]
    \end{align*}
        

    therefore,
    \begin{align*}
        \langle \hat n_k \rangle &= \frac{1}{Z_G}\Tr{e^{-\beta \hat K_0}e^{\beta \hat K_0}\hat a_ke^{-\beta \hat K_0}\hat a_k^{\dagger}}
        = \frac{1}{Z_G}\Tr{e^{-\beta \hat K_0} \hat a_k(\beta\hbar)\hat a_k^{\dagger}}\\
        & = \frac{1}{Z_G}\Tr{e^{-\beta \hat K_0} (1\pm \hat a_k^{\dagger}\hat a_k(\beta\hbar))}\\
        & = \frac{1}{Z_G}\Tr{e^{-\beta \hat K_0}} \pm \frac{1}{Z_G}\Tr{e^{-\beta \hat K_0}\hat a_k^{\dagger}\hat a_k(\beta\hbar)}\\
        & = 
    \end{align*}
    Using the Baker-Campbell-Hausdorff formula:
    \begin{equation*}
        e^{\beta \hat K_0}\hat a_ke^{-\beta \hat K_0} = \hat a_k + [e^{\beta \hat K_0}, \hat a_k] 
        + \frac{1}{2!}[e^{\beta \hat K_0}, [e^{\beta \hat K_0},\hat a_k]]
        + \frac{1}{3!}[e^{\beta \hat K_0}, [e^{\beta \hat K_0}, [e^{\beta \hat K_0},\hat a_k]]]
        + \cdots
    \end{equation*}
    
    \item Calculate $\langle \hat N \rangle$ for bosonic particles!
    \item Do something
    \item asdfghjk \begin{equation*}
        \circled{$!$}, \circled{$\nabla$}, \mathcal G(x, \tau, x', \tau')
    \end{equation*}

\end{enumerate}




\end{document}